
\begin{figure}[t]
    \centering

    \subfloat[
        Using purely structural constructs to create MUXes can provide long and superfluous code.
    ]{
        \begin{minipage}{0.8\textwidth}
            \footnotesize
            \inputminted[frame=single]{systemverilog}{media/code/c-like/low.svh}
            \label{subfig:structural_only}
        \end{minipage}
    }

    \subfloat[
        Using C-like constructs such as a \mintinline{systemverilog}{function}, \mintinline{systemverilog}{if} statement, and \mintinline{systemverilog}{for} loop can provide much cleaner code.
    ]{
        \begin{minipage}{0.8\textwidth}
            \footnotesize
            \inputminted[frame=single]{systemverilog}{media/code/c-like/high.svh}
            \label{subfig:c-like_allowed}
        \end{minipage}
    }

    \caption[
        Structural vs. C-like Verilog
    ]{
        Provided is an example of when C-like constructs can be used to write cleaner code compared to purely structural constructs. Sub-figures \ref{subfig:structural_only} and \ref{subfig:c-like_allowed} both implement the Find First Set operation, but \ref{subfig:c-like_allowed} is better.
    }
    \label{fig:c-like}

\end{figure}
