
\begin{figure}[t]
    \centering

    \subfloat[
        Using purely structural constructs to create MUXes can provide long and superfluous code.
    ]{
        \begin{minipage}{0.8\textwidth}
            \footnotesize
            \inputminted[frame=single]{systemverilog}{media/code/c-like/low.svh}
        \end{minipage}
    }

    \subfloat[
        Using C-like constructs such as a \mintinline{systemverilog}{function}, \mintinline{systemverilog}{if} statement, and \mintinline{systemverilog}{for} loop can provide much cleaner code.
    ]{
        \begin{minipage}{0.8\textwidth}
            \footnotesize
            \inputminted[frame=single]{systemverilog}{media/code/c-like/high.svh}
        \end{minipage}
    }

    \caption{Comparison of purely structural Verilog versus C-like Verilog. To demonstrate this comparison, provided are two different implementations of the Find First Set operation.}
    \label{fig:c-like}

\end{figure}
