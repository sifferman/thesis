%
%  Abstract
%

\begin{abstract}
\addcontentsline{toc}{chapter}{Abstract}

In the rapidly expanding semiconductor industry, there is an increasing demand for skilled chip developers. Yet, the steep learning curve associated with Hardware Description Languages (HDLs), such as Verilog, often acts as a significant barrier for students hoping to pursue a career in digital design. Drawing upon my experience as an educator, which includes teaching Verilog to UCSB's IEEE student chapter and serving as a Teaching Assistant for UCSB's Verilog courses, I have meticulously developed and refined a comprehensive set of methods and resources for Verilog education. My objective encompassed two key facets: equipping students with quality industry-preparation and kindling passion for exploring hardware design. Through a strategic blend of approaches consisting of the integration of accessible open-source tools, the enforcement of popular coding style guides, the implementation of autograders for personalized feedback, and the incorporation of open-source IP blocks into lessons, students can attain proficiency in designing RTL (Register Transfer Level) for rigorously verified hardware systems. These strategies help reduce Verilog's steep learning curve while also expediting the introduction of more advanced topics in digital design and computer architecture. The methods and resources detailed in this thesis will prepare students for the expectations of the semiconductor industry, enhance their coding skills, and promote an accessible and engaging learning environment, ultimately meeting the growing demand for chip developers.

\end{abstract}
