
\chapter{Conclusion}
\label{chapter:Conclusion}

Universities should be working to lower the barrier of entry into SystemVerilog design. Throughout this thesis, several critical challenges in Verilog/SystemVerilog education have been addressed and resolved, contributing to the enhancement of the learning experience for students. These issues encompassed a range of areas, and the solutions put forward have had a significant impact.

\begin{itemize}
    \item \textbf{Synthesizable vs. Verification Features:} A critical issue in Verilog education lies in distinguishing between synthesizable and verification features. This thesis has addressed this concern through multiple avenues. Using netlist graph viewers like DigitalJS Online, which enables students to visualize the synthesis process, helps visually demonstrate what constructs lead to valid netlists. Additionally, style guides like from the lowRISC Organization act as the best practices for writing synthesizable code. Finally, autograders can provide immediate, personalized feedback to students on whether their code is both behaviorally correct and synthesizable.
    \item \textbf{Prevalence of Bugs in Common HDL Tools:} Common Verilog and SystemVerilog tools often suffer from bugs when using lesser-used features. To mitigate this, style guides by lowRISC, Bespoke Silicon Group, and more should be used to teach students the best syntax and strategies for avoiding common pitfalls. Autograders can also play a pivotal role by seamlessly testing student code across a multitude of tools. If a submission passes tests for several tools, it is much more likely that it will work properly for all tools.
    \item \textbf{Inaccessibility of Proprietary Tools:} The reliance on proprietary tools in Verilog education has been a barrier to accessibility for many beginners. Open-source tools highlighted in this work present a solution by being cost-free, easier to install, and more user-friendly. This shift towards open-source tools enhances equity in education and industry and enables a broader spectrum of engineers to engage effectively with Verilog.
    \item \textbf{The Lack of Reliable Educational Resources:} There is a shortage of reliable educational resources in Verilog/SystemVerilog. In this thesis, a comprehensive list of resources has been provided in Section IV, serving as a valuable reference for both instructors and students. These resources cover a range of use-cases and are designed to support a deeper understanding of Verilog.
    \item \textbf{Interdisciplinary Nature of Chip Design:} Verilog and SystemVerilog education must acknowledge the interdisciplinary nature of chip design, involving both software and hardware components. This thesis emphasizes the importance of dedicating time to teaching essential skills like Git version control. Furthermore, I have described that DigitalJS Online should be used to create a stronger connection between Verilog code and hardware circuits, helping both software and hardware engineers become more well-rounded.
\end{itemize}

In light of the rapidly evolving landscape of open-source hardware, there is a pressing need for more universities and educators to modernize their digital design and computer architecture curriculums. Students deserve a curriculum that not only imparts technical knowledge but also inspires them to explore the field further. By incorporating elements like open-source tools, hands-on projects, and interactive platforms, educators can make Verilog education more engaging and appealing to a broader range of students. Furthermore, as demonstrated by the inclusion of ``for-fun'' leaderboards and ``Ungrading'' with autograders, fostering a collaborative and vibrant environment can help create a more valuable and enthusiastic learning experience. Institutions should be urged to reevaluate and revamp their Verilog courses to provide students with a more exciting and fulfilling educational experience, ultimately preparing them for the challenges and opportunities in the ever-evolving world of digital design and computer architecture.
