
\chapter{Potential Applications in Other Classes}
\label{chapter:other_classes}

So far, this thesis has described approaches that could significantly benefit courses in introductory Verilog, digital design, and computer architecture. However, Verilog education stands to gain much more from open-source and publicly available resources. In this short section, I will summarize potential ways that open-source resources could be used in other advanced courses.

\section{``Verification with UVM and SVA''}

Verification is an enormous aspect of chip design, so teaching students the principles of Universal Verification Methodology (UVM) and SystemVerilog Assertions (SVA) can be of paramount value in industry preparation. Courses like North Carolina State University's ECE 748: ``Advanced Verification with UVM'' have seen large popularity as companies are in constant demand for well-trained verification engineers. As of 9/17/23, Verilator has limited support for UVM and SVA, but is getting closer to full support every day \cite{VerilatorUVM, BieganskiORConf, VerilatorSVA}. If complete support is required, the open-source build-manager FuseSoC can still provide an accessible CLI for proprietary tools, lowering the learning curve. Additionally, adopting a comprehensive Design Verification (DV) style guide can ensure that students continue following best practices when working on verification tasks. Notably, lowRISC has a popular and thorough UVM and SystemVerilog DV feature style-guide \cite{lowRISCstyleguides}.

\section{``Embedded Systems and SoC Design''}

Embedded systems and SoC design courses can leverage a plethora of open-source, high-speed IP blocks that are commonly used in FPGA designs. There are many popular open-source designs for HDMI \cite{hdlutilhdmiGitHub, projfdisplaycontrollerGitHub, cliffordwolfSimpleVOutGitHub}, Ethernet \cite{alexforencichverilogethernetGitHub}, PCIe \cite{alexforencichverilogpcieGitHub, enjoydigitallitepcieGitHub}, and AXI \cite{pulpplatformaxiGitHub, alexforencichverilogaxiGitHub}. An educator could provide a similar experience to my ``Labs with CVA6'' project by teaching students the inner workings of advanced serial communication modules. Proficiency in high-speed interfaces is highly sought after in industry positions.

\section{``ASIC and VLSI Projects''}

For courses focusing on Application-Specific Integrated Circuits (ASICs), open-source resources become critical. Licensing and signing Non-Disclosure Agreements (NDAs) for proprietary PDKs are often impractical or time-consuming for instructors, limiting course opportunities. Fortunately, initiatives like the OpenROAD Project and the SkyWater PDK (SKY130) offer students access to fully-featured ASIC flows. For example, UC Berkeley's EE 194: ``The Tapeout Class'' utilized Hammer and OpenROAD to offer students the opportunity to tape out an SoC in a semester \cite{ZhaoLatchUp}. Moreover, affordable SKY130 fabrication opportunities like Tinytapeout (\SI{160}{\micro\metre} \texttimes \SI{100}{\micro\metre} for \$100 \cite{tinytapeoutTinyTapeout}) and Google MPW lottery (\SI{2920}{\micro\metre} \texttimes \SI{3520}{\micro\metre} for free \cite{efablessCaravel}) enable students to take their designs from simulation to real-world fabrication, providing a hands-on experience that reinforces their understanding of the ASIC design process.
